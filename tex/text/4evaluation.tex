\section{제출 형태 및 채점 기준}{\label{sec:eval}}

\subsection{제출 형태}
아래의 파일들을 \textcolor{RubineRed}{PJ\_학번.zip} 으로 압축하여 제출한다. (예: PJ\_20181485.zip)
\\ \textcolor{Red}{파일명 및 확장자가 잘못될 경우 감점 있음}
\begin{enumerate}\tightlist
    \item Train에 사용된 주피터 노트북 : \textcolor{RubineRed}{train\_학번.ipynb}
    \begin{itemize}
        \item Train 시 할당받은 GPU의 종류를 확인할 수 있도록
        \href{https://github.com/jaehyun-ko/EEE4178-Project-2022/blob/main/src/Project_2022_utils.ipynb}{프로젝트 도구}에서 제시한 방법대로
        \textcolor{Orange}{!nvidia-smi} 셀 및 출력을 포함한다.
        \item Train의 수행 시간 또한 측정하여 출력한다.
    \end{itemize}
    
    \item Train 후 저장된 model : \textcolor{RubineRed}{학번.pth}
    \item Test에 사용할 파이썬 스크립트 : \textcolor{RubineRed}{test\_학번.py}
    \\train 을 통해 저장한 모델을 불러와서 동작하도록 작성하되, Test set이 주어져 있지 않으므로 Validation set을 test하도록 작성한다.
    \item 보고서 : \textcolor{RubineRed}{project\_학번.pdf}
        \begin{itemize}
            \item 보고서는 다음 내용을 포함한다.
            \begin{enumerate}[1.]
                \item 과제 목표
                \item 배경 이론
                \item 과제 수행 방법
                \item 결과 및 토의
                \\ 할당받은 GPU 종류 및 학습에 사용된 시간을 기재한다. 예를 들어 \textcolor{Orange}{Tesla T4 환경에서 2.003초}
                % 이거 flops 계산으로 변경 해 주세요!!
                \item 참고 문헌
            \end{enumerate}
        \end{itemize}
\end{enumerate}

\subsection{채점 기준}
    \begin{itemize}\tightlist
        \item Model Accuracy (순위) : 50\%
        \item 프로젝트 보고서 : 50\%
    \end{itemize}


