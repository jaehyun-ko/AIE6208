\section{프로젝트 일정 및 제한 사항}{\label{sec:schedule}}

\subsection{제출 기한}
\begin{itemize}\tightlist
    \item 제출 마감 : 12월 18일 (일) 자정(23:59), 지각 제출에 대한 감점 있음
    \item 지각 제출 기한 : 12월 20일 (화) 자정 (23:59), 일당 5점씩 감점, 이후 제출 불가
\end{itemize}


\subsection{제한 사항}
    \begin{itemize}\tightlist
            \item 모델의 학습 관련
            \begin{itemize}\tightlist
                \item 모델은 Google Colab에서 제공하는 GPU 기준 \textcolor{red}{10분 이내에 학습 완료}되어야 한다.
                \item Train 시간이 10분 이상일 시 초과 시간에 따라 감점한다.
            \end{itemize}
        \item 모델 관련
            \begin{itemize}\tightlist
                \color{red}
                \item Pretrained model은 사용 불가능하다.
                \item 모델 구조는 분류 분야의 논문을 참고해서 구성해도 좋으나, torch 또는 torchvision 등을 통해 불러오지 않고 직접 구현해야 한다. 
                \item Torch에서 (대체로 Pretrained model과 함께) 제공하는 ResNet이나 MobileNet, Transformer 따위의 모델을 불러와서 사용하는 것은 불가능하다.
            \end{itemize}
        \item 라이브러리 관련
            \begin{itemize}\tightlist
                \item Python 표준 라이브러리, Pytorch, Numpy 제공 라이브러리만 모델 구성 및 학습에 사용 가능하다.
                \item Matplotlib, Seaborn 등 시각화를 위한 라이브러리는 자유롭게 활용해도 좋다.
                \item 언급된 것 외의 학습성능 향상을 위한 라이브러리 사용 시 조교에게 문의하여라.
            \end{itemize}
        \item 데이터 관련
            \begin{itemize}\tightlist
                \item 데이터 전처리는 자유롭게 진행할 수 있다.
                \item 데이터로더는 \href{https://github.com/jaehyun-ko/EEE4178-Project-2022/blob/main/src/Project_2022_utils.ipynb}{프로젝트 도구}에서 제공된 것을 가공하지 않고 사용한다.
            \end{itemize}
    \end{itemize}
