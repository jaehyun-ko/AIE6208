\section{논문 요약}{\label{sec:review}}

\subsection{Multi-modal fusion transformer for end-to-end autonomous driving}{\label{subsec:Transfuser}}
\begin{figure}[htp]
    \centering
    \includegraphics[width=0.8\textwidth]{figures/Transfuser.png}
    \caption{논문에서 해결하려는 문제 상황}
    \label{fig:tfprob}
\end{figure}
이 논문은 \ref{fig:tfprob} 의 상황처럼
라이다(LiDAR : Light Detection And Ranging) 센서로 얻을 수 있는 주변의 차량의 위치에 따른 교통정보와
카메라로 얻을 수 있는 신호기에 따른 교통정보가 다른 경우,
두 센서로부터 얻을 수 있는 정보를 결합하여 차량의 주행을 제어하는 것을 목적으로 한다.


\subsection{YOLOv7: Trainable bag-of-freebies sets new state-of-the-art for real-time object detectors}{\label{subsec:yolov7}}

asdf