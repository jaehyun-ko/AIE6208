\section{개요}{\label{sec:intro}}

\subsection{목적}
\begin{itemize}\tightlist
    \item 최근 출시되는 자율주행 자동차에 포함된 pattern recognition/computer vision 관련 요소기술을 조사한다.
    \item 해당 요소기술과 수업에서 다룬 주제들 간의 연관성을 파악한다.
\end{itemize}

\subsection{선정 모델 및 브랜드}
선정한 모델은 \textbf{Tesla Model S} 이다.
테슬라가 오토파일럿 기능을 앞세워 자율주행 시장을 선도하고 있기 때문이다.
해당 모델에 적용된 pattern recognition/computer vision 관련 요소기술은 다음과 같다.
\begin{itemize}\tightlist
    \item \textbf{Autopilot}: 자율주행 기능
    \item \textbf{Autopark}: 주차 자동화 기능
    \item \textbf{Autosteer}: 자동 조향 기능
\end{itemize}

\subsection{선정 논문}
\cite[Multi-modal fusion transformer for end-to-end autonomous driving]{Transfuser} 및 
\cite[YOLOv7: Trainable bag-of-freebies sets new state-of-the-art for real-time object detectors]{YOLOv7}
논문을 선정하였다.
첫 번째 논문을 선정하게 된 이유는 자율주행 시스템의 구성요소 중 하나인 \textbf{perception}에 대한 연구이면서,
필자의 연구 분야인 \textbf{multi-modal fusion}에 대한 연구이기 때문이다.
두 번째 논문을 선정하게 된 이유는 자율주행 시스템의 구성요소 중 하나인 \textbf{perception}에 대한 연구이면서,
영상처리 전반에 대한 논의를 하고 있기 때문이다.

